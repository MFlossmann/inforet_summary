\documentclass[a4paper]{scrartcl}

% \usepackage{amsmath}
\usepackage[english]{babel}
% \usepackage{lmodern}
\usepackage{anyfontsize}
\usepackage{listings}
\usepackage{amssymb}% http://ctan.org/pkg/amssymb
\usepackage{pifont}% http://ctan.org/pkg/pifont
\newcommand{\cmark}{\ding{51}}%
\newcommand{\xmark}{\ding{55}}%
\usepackage{color}

\definecolor{mygreen}{rgb}{0,0.6,0}
\definecolor{mygray}{rgb}{0.5,0.5,0.5}
\definecolor{mymauve}{rgb}{0.58,0,0.82}
\lstset{ %
  backgroundcolor=\color{white},   % choose the background color; you must add \usepackage{color} or \usepackage{xcolor}; should come as last argument
  basicstyle=\footnotesize,        % the size of the fonts that are used for the code
  breakatwhitespace=false,         % sets if automatic breaks should only happen at whitespace
  breaklines=true,                 % sets automatic line breaking
  captionpos=b,                    % sets the caption-position to bottom
  commentstyle=\color{mygreen},    % comment style
  % deletekeywords={...},            % if you want to delete keywords from the given language
  escapeinside={\%*}{*)},          % if you want to add LaTeX within your code
  extendedchars=true,              % lets you use non-ASCII characters; for 8-bits encodings only, does not work with UTF-8
  frame=single,	                   % adds a frame around the code
  keepspaces=true,                 % keeps spaces in text, useful for keeping indentation of code (possibly needs columns=flexible)
  keywordstyle=\color{blue},       % keyword style
  language=Python,                 % the language of the code
  morekeywords={*,...},            % if you want to add more keywords to the set
  % numbers=left,                    % where to put the line-numbers; possible values are (none, left, right)
  % numbersep=5pt,                   % how far the line-numbers are from the code
  % numberstyle=\tiny\color{mygray}, % the style that is used for the line-numbers
  rulecolor=\color{black},         % if not set, the frame-color may be changed on line-breaks within not-black text (e.g. comments (green here))
  showspaces=false,                % show spaces everywhere adding particular underscores; it overrides 'showstringspaces'
  showstringspaces=false,          % underline spaces within strings only
  % showtabs=true,                  % show tabs within strings adding particular underscores
  % stepnumber=2,                    % the step between two line-numbers. If it's 1, each line will be numbered
  stringstyle=\color{mymauve},     % string literal style
  tabsize=4,	                   % sets default tabsize to 2 spaces
  title=\lstname                   % show the filename of files included with \lstinputlisting; also try caption instead of title
}

\begin{document}
\section{Inverted Index}
\paragraph{Idea:} For each word, pre-compute and store a \emph{sorted} list of
ids of documents containing them.

$\Rightarrow$ Inverted Lists:
\begin{table}[!htbp]n
  \centering
  \caption{Example of inverted lists}
  \label{tab:inverted_list}
  \begin{tabular}{ll}
  astronauts& 13, 57, 64, 77, 104, ... \\
  moon & 5, 23, 57, 63, 104, 257, ...
  \end{tabular}
\end{table}

\lstinputlisting[caption=Intersecting sorted lists in linear
time]{code/intersect.py}
\paragraph{Querying with more than 2 keywords:}
\begin{enumerate}
\item Intersect $L_1$ and $L_2\rightarrow L_{12}$
\item Intersect $L_{12}$ and $L_3\rightarrow L_{123}$
\item ...
\end{enumerate}
Possible Optimizations:
\begin{itemize}
\item Order the lists s.t. $|L_1|\le|L_2|\le|L_3|\le...\le|L_k|$
\end{itemize}

\section{Ranking}
Ways to rank the relevance of the found entries with respect to the query.
\begin{itemize}
\item In the inverted lists, each doc id also has a \emph{score}
  \begin{table}[!htbp]
    \centering
    \caption{Scoring}
    \label{tab:scored_list}
    \begin{tabular}{ll}
      astronauts& 13:0.2, 57:0.5, 64:0.3, 77:0.25, 104:0.1, ... \\
      moon & 5:0.3, 23:0.6, 57:0.8, 63:0.2, 104:0.1, 257:0.9, ...
    \end{tabular}
  \end{table}
\item When merging, aggregate the scores and then sort.
\item Doc ID and score (or more info): \emph{posting}
\item Sorting only the top-$k$ hits takes $\Theta(n)$ time for constant $k$.
  (with heapsort)
\end{itemize}

\subsection{Scoring algorithms}
\label{sec:scoring_algorithms}
\begin{itemize}
\item \emph{Term frequency} (tf): Number of times a word occurs in a document
  \begin{itemize}
  \item Problem: Words like of, a, and, the,...
  \item Just ignore these
  \end{itemize}
\item \emph{Inverse document frequency} (idf):
  \begin{equation}
    \mathrm{idf}=\log_2\frac{N}{\mathrm{df}}
  \end{equation}
  $N$: Total number of documents, $\mathrm{df}$: Document frequency (number of
  documents containing a particular word)
\item Combining tf and idf: good way to decrease the influence of “of”, “the”,
  “and”, ...

  Problems with tf.idf:
  \begin{itemize}
  \item \textbf{Problem 1:} Longer documents are automatically weighted higher.
  \item \textbf{Problem 2:} Double tf doesn't mean it's doubly “about“ the query.
  \end{itemize}
\item \emph{BM25}
  \begin{equation}
    \mathrm{BM25}=\mathrm{tf}^*\cdot \log_2\frac{N}{\mathrm{df}}
  \end{equation}
  \begin{equation}
    \mathrm{tf}^*=\mathrm{tf}\cdot \frac{(k+1)}{k\cdot(1-b + b \cdot \frac{\mathrm{DL}}{\mathrm{AVDL}})}
  \end{equation}
  DL: Document length, AVDL: average document length. $k,b$: Magic numbers (good
  setting: $k=1.75, b=0.75$)
  \begin{itemize}
  \item Normal tf.idf: $k=\infty, b=0$
  \end{itemize}
\end{itemize}

\subsection{Evaluation}
\subsubsection{Precision}
\begin{itemize}
\item Precision (\emph{P@k}): Percentage of relevant documents among the top $k$
  \begin{itemize}
  \item Example:\\
    \begin{tabular}{ll}
      Query:& matrix movies \\
      Relevant:& 10, 582, 877, 10003 \\
      Result list:& 582(\cmark), 17(\xmark), 5777(\xmark), 10003(\cmark), 10(\cmark), 37(\xmark),...\\
      \emph{P@1}:&100\% \\
      \emph{P@2}:&50\% \\
      \emph{P@3}:&33\% \\
      \emph{P@4}:&50\% \\
      \emph{P@5}:&60\% \\
    \end{tabular}
  \end{itemize}
\item \emph{P@R}: \emph{P@k}, where $k=$amount of relevant documents
\item \emph{Average Precision} (AP)
  \begin{itemize}
  \item $R_1,R_2,...,R_k$: Sorted list of positions of the relevant documents in
    the result list. \\
    \begin{tabular}{ll}
      Query:& matrix movies \\
      Relevant:& 10, 582, 877, 10003 \\
      Result list:& 582(\cmark), 17(\xmark), 5777(\xmark), 10003(\cmark), 10(\cmark), 37(\xmark),...,877(\cmark) \\
      $R_1,...,R_4$:& 1,4,5,40 \\
      $P@R_1$,...,$P@R_4$:& 100\%, 50\%, 60\%, 10\% \\
      AP:& (100\% + 50\%, 60\%, 10\%)/4 = 55\%
    \end{tabular}
  \end{itemize}
\end{itemize}

\subsubsection{Mean precisions}
Measure quality of a system by taking the mean of a given measure over all
queries:
\begin{itemize}
\item \emph{MP@k}: mean of the P@k values over all queries
\item \emph{MP@R}: mean of the P@R values over all queries
\item \emph{MAP}: mean of the AP values over all queries
\end{itemize}

\subsubsection{Discounted cumulative gain (DCG, nDCG)}
Relevance not binary (i.e.: 0: not relevant, 1: somewhat relevant, 2: very
relevant)
\begin{itemize}
\item \emph{Cumulative gain:} \[CG@k=\sum_{i=1,...,k} \mathrm{rel}_i\]
\item \emph{Discounted CG:} \[DCG@k = \mathrm{rel}_i +
  \sum_{i=2,...,k}\frac{\mathrm{rel}_i}{\log_2 i}\]
\item Normalize by maximally achievable value:
\item \emph{Ideal DCG}: \[iDCG@k=DCG@k\ \mathrm{of\ ideal\ ranking}\]
\item \emph{Normalized DCG}: \[nDCG@k = \frac{DCG@k}{iDCG@k}\]
\end{itemize}

\subsubsection{Binary preference (bpref)}
If you have way more unjudged documents than judged ones
\[\mathrm{bpref}=\frac{1}{|R|}\cdot\sum_{r\in RR}\frac{1-|NR(r)|}{\min(|R|,|N|)}\]
\begin{itemize}
\item $R$: Set of all judged relevant docs
\item $N$: Set of all judged not relevant docs
\item $RR:$ Docs in $R$ and in result list
\item $NR(r\in RR):$ docs from the $|R|$ top-ranked from $N$ ranked before $r$
\item Example: Result list with 7 judged documents overall\\ 
    \begin{tabular}{ll|}
      \#1&judged relevant\\
      \#2&not judged\\
      \#3&not judged\\
      \#4&judged not relevant\\
      \#5&judged relevant\\
    \end{tabular}
    Not in result list:
    \begin{tabular}{l}
      $A$, judged relevant \\
      $X, Y, Z$, judged not relevant
    \end{tabular}
    \begin{itemize}
    \item $R=\{\#1, \#5, A\} $
    \item $N=\{\#4, X, Y, Z\}$
    \item $RR=\{\#1,\#5\}$
    \item $NR(\#1)= \{\}, |NR(\#1)|=0$
    \item $NR(\#5)= \{\#4\}, |NR(\#5)|=1$
\[\Rightarrow\mathrm{bpref} = \frac{1}{3}\left( \left( 1-\frac{0}{3} \right) +
        \left( 1-\frac{1}{3} \right) \right) = \frac{5}{9}\]
    \end{itemize}
\end{itemize}

\section{Efficient List Intersection}

\end{document}

%%% Local Variables:
%%% coding: utf-8
%%% mode: latex
%%% TeX-engine: xetex
%%% TeX-master: t
%%% End:
